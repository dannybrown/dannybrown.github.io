\documentclass[10pt,letterpaper]{article}
\usepackage[letterpaper,margin=0.6in]{geometry}
\usepackage{mdwlist}
\usepackage[T1]{fontenc}
\usepackage{textcomp}
\usepackage{titlesec}
\pagestyle{empty}
\setlength{\tabcolsep}{0em}
\titlespacing\section{0pt}{12pt plus 4pt minus 2pt}{0pt plus 2pt minus 2pt}
\usepackage{hyperref}

\begin{document}
	\noindent {\Large\bfseries Daniel Brown BSc (Hons) MSc MBCS}

	\noindent Imagine Cup World Finalist / Open Source Developer / Blogger / Private Pilot

	\noindent\makebox[\linewidth]{\rule{\textwidth}{0.4pt}}
	
	\section*{Experience}
	\subsection*{Founder and Chief Engineer, Brown Enterprises \small{(June 2020 - Current)}}
	Started my own company to provide tech leadership consulting, software development services, technical due diligence and interim/fractional CTO services. First development contract was with Third Space Learning to provide an instant messaging client for use by tutors to aid children with learning maths from home during the coronavirus pandemic. Brown Enterprises also develops its own products, the first of which was a "Track and Trace" system for use by small hospitality businesses. Fortunately this was superseded by the NHS system before commercial use. Brown Enterprises has also provided technical advice and roadmapping to pre-money start-ups.
	
	\subsection*{Chief Technology Officer, Defty \small{(October 2018 - June 2020)}}
	Employee number 1 at Defty, a Website Builder and Domain Name Management System for small businesses and individuals. 
	Hired the complete engineering and UX/UI team. Developed technical strategy and quality assurance processes as well as being involved in hands-on development. Mentored Junior Engineers and handled day-to-day people management. Built static website generator, Extensible Provisioning Protocol (EPP) client and server, authentication system, analytics system and designed overall system architecture. Fostered a unique company culture and processes to put quality front and centre of everything Defty did. Involved in product and UI/UX decisions at every stage. Read more at \href{https://dannybrown.blog/2020/06/01/defty-the-easy-to-use-website-builder/}{https://dannybrown.blog/2020/06/01/defty-the-easy-to-use-website-builder/}
	
	\subsection*{Senior Frontend Engineer, DriveTribe \small{(July 2018 - September 2018)}}
	Migrated codebase from using a proprietary in-house CI system to CircleCI to reduce internal workload and enable adoption of a full Continuous Deployment workflow. Wrote admin tool and made changes to front-end to facilitate the expansion of the site into the Australian and Chinese markets (including workarounds for the Great Firewall). Developed "MyGarage" -- a feature which allows users to upload photos of and information about their past, present and dream vehicles for others to comment on, which was announced by Clarkson, Hammond and May of Top Gear fame. Developed on an isomorphic application using React \& Styled Components.
	
	\subsection*{Lead Software Engineer, PepperHQ \small{(September 2017 - July 2018)}}
	Promoted to reflect the increased responsibilities I had taken on including technical recruitment, mentoring junior engineers and improving technical processes. Improvements to technical processes included a transition to a culture of pull requests and code reviews and the introduction of end-to-end testing for the API server, these changes resulted in an improvement to team velocity. Introduced Code-defined infrastructure which enabled the team to deploy more often and treat servers as cattle not pets. Led efforts to scale platform from fifty locations to several hundred, allowing Pepper to onboard more paying customers. Rebuilt strained ordering system to be independently scalable and more performant. Heavily involved in shaping technical strategy.
	
	\subsection*{Senior Software Engineer, PepperHQ \small{(September 2016 - September 2017)}}
	Responsible for the Pepper Platform, which enabled mobile clients and third party point of sale systems to place orders at coffee shops and food chains, users to accrue loyalty perks and merchants to keep in contact with their customers and issue vouchers. The platform consisted of a monolithic node.js express application providing a REST API and AWS Lambda Functions for data processing -- all backed by MongoDB and DynamoDB databases. Scaled service to over 2 million HTTP transactions a week. Implemented CI/CD through Codeship. Introduced Linting to all JS codebases. Designed and developed Pepper Vouchers and a replacement for the existing loyalty system which allowed for new types of loyalty schemes including pseudo currency (money equivalents to spend in store) and product specific stamp cards.
	
	\subsection*{Dashboard Software Engineer, Tapdaq \small{(April 2016 - September 2016)}}
	Worked on delivering a new React based front-end for the Tapdaq consumer facing dashboard. Maintained existing Angular.js dashboard during phase out. Developed using an agile SCRUM methodology. Set up CDN to improve response times for people receiving ads.
	
	\subsection*{Software Engineer, Trainline \small{(September 2015 - April 2016)}}
	Joined Trainline as a software engineer in my first full-time role after my MSc. Designed, implemented, tested and deployed a RESTful API in Node.js to allow for personalised train ticket deals to be shown to users based on their geolocation. Remediated a large C\# web application and several mission critical Windows Services to be used in the Amazon Web Services cloud. Contributed to the new React.js based homepage. Outside of development I also attended RailTech hackathons as a Trainline employee and attended recruitment events such as Silicon Milkroundabout.
	
	\subsection*{Software Engineer, Ripple, Microsoft Imagine Cup World Finals 2014}
	Sole software engineer for the UK team in the world final of the prestigious Microsoft Imagine Cup. Around 60,000 people in teams of 3-4 applied to compete and we were fortunate enough to compete against the top 10 teams in our chosen category: Innovation. I was selected as the Software Engineer by Microsoft UK due to my eagerness to take part and their previous positive experience of working with me on App Builder Rewards in which I showed not only technical acumen but the ability to communicate with non-technical people -- important in a competition in which the business pitch is given equal weight to the technical solution.

	Ripple, a social network which allowed users to communicate with people located nearby and literally expand their sphere of influence based on a reputation system, was developed using Azure Mobile Services (hosted node.js and MSSQL) and included a C\#/XAML/MVVM Windows Phone application. Being the only developer meant that I tackled the entire application lifecycle from specification capture through to deployment, maintenance and user support.

	\subsection*{Full-Stack Web Developer, Microsoft Corporation \small{(February 2013 - September 2013)}}
	App Builder Rewards was an effort by Microsoft to promote the development of Windows 8 and Windows Phone applications. I developed an e-commerce style website which allowed developers to use their Windows Store submissions as currency to pay for prizes such as gadgets and experience days. The site was developed in PHP -- which I had an extensive knowledge of having recently developed an IDE for the language as part of my final year project -- and utilised a MSSQL database, both hosted on Windows Azure. The site itself was a success, granting prizes to thousands of developers and spurring interest in the Windows platforms.
	\newline \newline
	\bfseries{ Other Experience: } \mdseries{Co-founder, QuickSync (Point of Sale/eBay/Amazon Stock Syndication Systems) \newline / Web Developer, HONEI, University of Hull / Computer Science Demonstrator, University of Hull}
	
	\section*{Education}
	\subsection*{MSc Advanced Computer Science, University of York \small{Distinction/Merit}}
	One of five students to receive a \pounds5000 scholarship out of eighty applicants to masters level computer science courses at The University of York. This scholarship was awarded based on undergraduate performance, commercial experience and passion for the subject. The course culminated in my research project in which I developed a model-driven software tool to Query and Analyse git repositories with an eye to allowing software engineers to make more informed choices about their development processes and learn more about their projects contributors. In the taught portion of the course I achieved a grade of 99\% in the Concurrent and Real-Time Programming module taught by Professor Andy Wellings -- the highest grade in the modules history.

	\subsection*{BSc (Hons) Computer Science, University of Hull \small{First Class, Departmental Award}}
	Awarded the Computer Science Departmental Award for achieving the highest grade of a graduating student in Computer Science at The University of Hull in 2014, having achieved a first class grade in every module throughout the three years. Submitted a paper on developing DollarIDE, an integrated development environment for PHP, as my final year dissertation. Particularly enjoyed modules on Distributed Systems Programming and Languages and Their Compilers.

	\section*{Open Source and Personal Projects}
	\subsection*{LibreOffice}
	My first experience of open source development was fixing bug \#67159 in LibreOffice -- the open source MS Office Competitor. Fixing this bug involved writing some office-suite standard keyboard bindings which were missing from LibreOffice and therefore disrupting some peoples workflows when switching from OpenOffice or MS Office. These keyboard bindings were for inserting hyperlinks into documents, presentations and spreadsheets.
	\subsection*{Computer Science Blogs \small{(csblogs.com)}}
	The Computer Science Blogs network is an initiative I co-founded with a friend at The University of Hull to try to get more undergraduate students to write about their work and what they have achieved -- not only as a form of self promotion but as a way to improve their writing skills in preparation for their dissertations. CSBlogs.com aggregates all the students blogs and allows them to be discovered. The software consists of a an HTTP/JSON API, an RSS/ATOM feed aggregator, and front ends for the web and android. I used the development of the first version of CSBlogs as a sandbox for learning node.js.
	
	\subsection*{EpsilonGit}
	EpsilonGit was the productised outcome of my masters research project. Users can query and analyse git repositories of any size as domain models inside Eclipse Epsilon using the expressive Epsilon Object Language. Complex reports, such as a list of authors ordered by total number of lines of code committed which haven’t been replaced by a different author, can be generated in a fraction of the time required by existing solutions. The work has been open sourced and is now maintained by the Eclipse Epsilon project. 

	\subsection*{Website \& Blog}
	In the first year of my undergraduate degree I started a blog to document interesting things I had learnt, achieved or built. I have continued to write entries throughout my career. As well as hosting my blog dannybrown.io also contains my portfolio and links to relevant profiles on other sites such as GitHub.
	
	\section*{Personal Details}
 	\noindent References are available upon request. \newline \newline
	\textbf{Website \& Blog:} dannybrown.io / \textbf{Email:} d.t.brown@outlook.com / \textbf{Telephone (m):} +447960229733 \newline \textbf{Driving Licence:} Full, Clean UK License / \textbf{Pilots Licence:} In Training (PPL)
\end{document}
